\documentclass[professionalFonts,aspectratio=169]{beamer}
\usepackage[latin1]{inputenc}
\usetheme{Madrid}
\usecolortheme{rose}
\usepackage{listings}
\usepackage{fourier}
\usepackage[scaled=0.8]{luximono}
\usepackage{helvet}
\usepackage{color}
\usepackage{ulem}

\title[Isolating State with Monads]{%
  Isolating State with Monads\\An Introduction to Reader, Writer and State
}
\author{Ben Kolera}
\institute{bfpg.org}
\date{April 23, 2013}

\lstdefinelanguage{Scala}{
    morekeywords={%
        abstract,case,catch,class,def,do,else,extends,%
        false,final,finally,for,forSome,if,implicit,import,lazy,%
        match,new,null,object,override,package,private,protected,%
        return,sealed,super,this,throw,trait,true,try,type,%
        val,var,while,with,yield},%
    sensitive,%
    morecomment=[l]//,%
    morecomment=[s]{/*}{*/},%
}[keywords,comments]%

\definecolor{dkgreen}{rgb}{0,0.6,0}
\definecolor{gray}{rgb}{0.5,0.5,0.5}
\definecolor{mauve}{rgb}{0.58,0,0.82}
 
% Default settings for code listings
\lstset{frame=tb,
  language=scala,
  aboveskip=3mm,
  belowskip=3mm,
  columns=[c]fixed,%
  numbers=none,
  keywordstyle=\color{blue},
  commentstyle=\color{dkgreen},
  stringstyle=\color{mauve},
  tabsize=2,
  basicstyle={\small\ttfamily},
  basewidth={0.55em, 0.50em},
}

\newenvironment{changemargin}[2]{% 
  \begin{list}{}{% 
    \setlength{\topsep}{0pt}% 
    \setlength{\leftmargin}{#1}% 
    \setlength{\rightmargin}{#2}% 
    \setlength{\listparindent}{\parindent}% 
    \setlength{\itemindent}{\parindent}% 
    \setlength{\parsep}{\parskip}% 
  }% 
  \item[]}{\end{list}} 

\begin{document}

\begin{frame}
\titlepage
\end{frame}

\section{Introduction}

\subsection{Me}
\begin{frame}{Me}

\begin{itemize}
\pause \item Scala and Perl Dev at iseek Communications
\pause \item Lots of diverse codebases coded by a small team
\pause \item Often have to modify old bitrotted code.
\pause \item \sout{Hate} Suck at writing tests
\pause \item Need to be able to easily reason about code and be confident in changes.
\pause \item Love how types can keep my refactorings honest and correct
\pause \item Slowly trying to convince my team that types are awesome
\end{itemize}
\end{frame}

\subsection{Goals of this Talk}
\begin{frame}{Goals of this Talk}

\begin{itemize}
  \pause \item To take you on a refactoring journey!
  \pause \item To show you all of the little baby steps on the path to ReaderWriterState.
  \pause \item To highlight reasons why unrestricted side effects are bad.
  \pause \item To show to that purifying code $\neq$ removing state.
  \pause \item To show that types and structures can be used to build and restrict computations as well as data.
  \pause \item To convince you that all of this isn't hard; its just (brain alteringly) different!
  \pause \item To convince you that these weird and wonderful changes actually make your code easier to reason about, not the other way around.
\end{itemize}

\end{frame}

\subsection{Scalaz}
\begin{frame}{Scalaz}
\begin{itemize}
  \pause \item Library with lots of FP goodies missing from the std lib.
  \pause \item Focus is on composable typeclass based programming.
  \pause \item Really awesome!
  \pause \item Fair few code committers are familiar BFPG faces. :)
  \pause \item Code from this talk depends on it.
\end{itemize}
\end{frame}

\subsection{Monads (in 2 minutes)}
\begin{frame}{Monad Typeclass/Interface}

\begin{itemize}
\pause \item Typeclass for a type constructor that is 'flatmappable'
  \begin{itemize}
    \item We write a monad for \texttt{Option[\textunderscore]}, not \texttt{Option[String]}
  \end{itemize}        
\pause \item The typeclass has two methods:
  \begin{itemize}
    \pause \item return: \texttt{5.point[Id]}
    \pause \item flatMap: \texttt{5.point[Id].flatMap( x => 6.point[Id] )}
  \end{itemize}
\pause \item We get the map method for free. 
  \begin{itemize}
    \pause \item \texttt{5.point[Id].map( \textunderscore \ + 2 )}
    \pause \item \texttt{5.point[Id].flatMap( x => 6.point[Id].map( \textunderscore \ + x ) )}
  \end{itemize}
\pause \item Lots of other machinery written around the typeclass.
   \begin{itemize}
     \pause \item Sequence is the most important for us tonight.
     \pause \item \texttt{(1 to 5).toList.map( \textunderscore.point[Id] ).sequenceU}
  \end{itemize}
\end{itemize}

\end{frame}

\begin{frame}{Why is this useful?}
\begin{itemize}
  \pause \item Each monad describes a different computational context.
  \pause \item We're extremely restricted in how we can interact with a context:
    \begin{itemize}
      \pause \item We can transform what is inside with map.
      \pause \item Join it to another context with flatmap.
    \end{itemize}
  \pause \item This is useful to describe/enforce sequential computations.
  \pause \item Each monad defines flatmap/join in its own special way:
    \begin{itemize}
      \pause \item We can make lots of different kinds of chained computations.
      \pause \item We're going to use different monads to serialize different kinds of state between our computations.
    \end{itemize}
\end{itemize}
\end{frame}

\subsection{Example Programming Problem}
\begin{frame}{Our (toy) problem}

Read XML from a file and:

\begin{itemize}
\pause \item Pretty print it to standard out \begin{itemize}
\item Configurable indentation step (2 spaces,4 spaces, tabs)
\end{itemize}
\pause \item Validating the XML and print warnings to stderr \begin{itemize}
\item We only have one rule: <msg> may only include text
\end{itemize}
\end{itemize}

\end{frame}

\begin{frame}{Example Execution}

\begin{block}{Input}
\lstinputlisting[frame=none]{test.xml}
\end{block}

\begin{block}{Output}
\lstinputlisting[frame=none]{test.output}
\end{block}

\end{frame}

\section{Our Imperative Beginning}

\begin{frame}{Our Imperative Beginning}
\end{frame}

\begin{frame}{Our Globals}

\begin{itemize}
\item Global configuration of the indentation character(s).
\item Mutable buffer of errors to append to in the program.
\item Mutable stack of element names to track what we've seen.
\end{itemize}

\lstinputlisting[language=Scala,firstline=30,lastline=35]{src/main/scala/00-Start.scala}

\end{frame}

\begin{frame}{Indenter}

\begin{itemize}
\item Takes an indent level and appends level amount of the indentSeq to the text.
\end{itemize}

\lstinputlisting[language=Scala,firstline=37,lastline=39]{src/main/scala/00-Start.scala}

\end{frame}

\begin{frame}{Verifier}

\begin{itemize}
\item Examines the new event and the top of the global stack.
\item Appends error to errors if get a ElemStart inside of <msg>
\end{itemize}

\lstinputlisting[language=Scala,firstline=41,lastline=48]{src/main/scala/00-Start.scala}

\end{frame}

\begin{frame}{Stream Processor}

\lstinputlisting[language=Scala,firstline=51,lastline=66]{src/main/scala/00-Start.scala}

\end{frame}

\begin{frame}{Output}

\lstinputlisting[language=Scala,firstline=67,lastline=69]{src/main/scala/00-Start.scala}

\begin{itemize}
\pause \item Bzzt! This has a subtle bug
\pause \item The stream processing is lazy and isn't evaluated till the last line.
\pause \item Thus there aren't any errors in the buffer when we print them.
\pause \item Our harmless-looking variable is already causing us troubles!
\pause \item Sure, here we don't need or gain benefit from the stream, but:
  \begin{itemize}
    \pause \item Some degree of laziness is inevitable in fp.
    \pause \item Laziness and unconstrained side-effects don't play nice.
    \pause \item Even if you discount laziness, non referentially transparent functions don't compose, which is the point of FP!
  \end{itemize}
\end{itemize}

\end{frame}

\section{Reader}

\begin{frame}{Death to the Global Config!}

Before we go crazy fixing our side effects...

\begin{itemize}
\item With or without FP, global config is a code smell. Lets fix it!
\pause \item In OO, we'd use DI. 
\pause \item In scala, one way to DI is to use curried functions.
\end{itemize}

\end{frame}

\subsection{Curried Functions}

\begin{frame}{Curried Functions}

\begin{itemize}
\item Just a function that takes a config and returns another function.
\pause \item We can store that configured function and call it later.
\end{itemize}

\pause

\begin{block}{Old Indenter}
\lstinputlisting[frame=none,language=Scala,firstline=37,lastline=39]{src/main/scala/00-Start.scala}
\end{block}

\begin{block}{New Indenter}
\lstinputlisting[frame=none,language=Scala,firstline=25,lastline=27]{src/main/scala/01-CurriedDI.scala}
\end{block}

\end{frame}

\begin{frame}{Change to the Stream Processor}

\begin{block}{Old Processor}
\lstinputlisting[frame=none,language=Scala,firstline=52,lastline=56]{src/main/scala/00-Start.scala}
\end{block}

\begin{block}{New Processor}
\lstinputlisting[frame=none,language=Scala,firstline=40,lastline=45]{src/main/scala/01-CurriedDI.scala}
\end{block}

\end{frame}

\begin{frame}{Wait! That sucks!}
\begin{itemize}
\pause \item We need to inject every function that needs the config!
\pause \item Also need to thread the config through every level, as well.
\pause \item This idea is the right direction, but the idea wont scale.
\end{itemize}
\end{frame}

\subsection{Reader Monad}

\begin{frame}{Introducing the Reader Monad}
\begin{itemize}
\pause \item Same concept as currying, but the other way around:
  \begin{itemize}
    \pause \item We create a function that is waiting for configuration.
  \end{itemize}
\pause \item Joining two readers will thread the conf between them.
\pause \item Only need to put the config in the top and the monad will do the rest.
\end{itemize}

\lstinputlisting[language=Scala,firstline=31,lastline=36]{src/main/scala/02-ReaderMonad.scala}

\end{frame}

\begin{frame}{List of Readers}

No change to the stream processor!

\lstinputlisting[language=Scala,firstline=48,lastline=62]{src/main/scala/02-ReaderMonad.scala}

\end{frame}

\begin{frame}{Injecting the config}

\lstinputlisting[language=Scala,firstline=65,lastline=65]{src/main/scala/02-ReaderMonad.scala}

\begin{itemize}
\pause \item Transform our \texttt{List[PpReader[String]]} to \texttt{PpReader[List[String]]}
\pause \item Inject config into our final program to get a \texttt{List[String]}
\pause \item We only have to thread in the config once.
\pause \item If our conf is an immutable structure, no one can change it on us.
\pause \item It's almost like we have a reusable Config => Output program.
\pause \item Almost: since there is state that is not contained in the program, this is impossible right now.
\pause \item This state is a pain! Lets move onto moving it into our monad.
\end{itemize}

\end{frame}

\section{Writer}

\subsection{Collecting two lists in our reader}

\begin{frame}{Death to the error buffer!}
\begin{itemize}
\item The mutable error buffer is easiest to fix next.
\pause \item What about putting something mutable into the config?
  \begin{itemize}
    \pause \item That's better than what we have, but not good enough!
    \pause \item If a config was reused for two different monad executions, state may leak across them.
    \pause \item We need to keep our state sealed completely inside the monad
  \end{itemize}
\pause \item We're already collecting a list of readers:
  \begin{itemize}
    \pause \item Why not collect errors and readers at the same time? 
  \end{itemize}
\end{itemize}
\end{frame}

\begin{frame}{Changes to Stream Processor - Fold Initialisation}

\lstinputlisting[language=Scala,firstline=54,lastline=57]{src/main/scala/03-CollectingErrorList.scala}

\begin{itemize}
\item verifyNewElement now returns a list instead of Unit.
\pause \item We now build a \texttt{PpReader[(List[String],List[String])]} 
  \begin{itemize}
    \item Aliased to \texttt{PpProgram}
  \end{itemize}
\pause \item We now fold over the stream instead of mapping
\end{itemize}


\end{frame}

\begin{frame}{Changes to Stream Processor - Fold Step}

\lstinputlisting[language=Scala,firstline=57,lastline=72]{src/main/scala/03-CollectingErrorList.scala}

\end{frame}

\begin{frame}{Changes to Output}

\lstinputlisting[language=Scala,firstline=74,lastline=76]{src/main/scala/03-CollectingErrorList.scala}

\begin{itemize}
\item No more sequencing since our fold is doing that for us.
\pause \item Get back a tuple of both output lists
\pause \item Need to remember to reverse the output list due to cons-ing
\end{itemize}

\end{frame}

\begin{frame}{Collecting two lists - Results}

\begin{itemize}
\item We moved our error list into the monad. Win!
\pause \item Each level awkwardly needs to explicitly know about and append the errors.
\pause \item This would be a real pain with lots of nesting to the code.
\end{itemize}

\end{frame}

\subsection{Writer monad}

\begin{frame}{Introducing the Writer Monad!}
\begin{itemize}
  \item \texttt{Writer[L,A]} where there must be a \texttt{Monoid[L]}
    \begin{itemize}
      \pause \item \texttt{List("msg").tell} to log and return unit
      \pause \item \texttt{5.set(List("msg"))} to log and return 5
      \pause \item Monoid.mappend used to append log state together
    \end{itemize}
  \pause \item We just log and forget! Just like a logger.warn("msg") !
\end{itemize}

\pause

\lstinputlisting[language=Scala,firstline=34,lastline=34]{src/main/scala/04-WriterMonad.scala}

\pause
\lstinputlisting[language=Scala,firstline=47,lastline=54]{src/main/scala/04-WriterMonad.scala}

\end{frame}

\begin{frame}{Changes to our stream fold}

\begin{itemize}
\item Abstract out the event pattern match into a separate function  
\end{itemize}

\pause

\lstinputlisting[language=Scala,firstline=56,lastline=67]{src/main/scala/04-WriterMonad.scala}

\end{frame}

\begin{frame}{Changes to our stream fold}

\begin{itemize}
\item For each event:
  \begin{itemize}
    \pause \item All we really want to do is append the new output string to the output.
    \pause \item To do this we need to join the inner and outer monads.
 \end{itemize}   
\end{itemize}

\pause

\lstinputlisting[language=Scala,firstline=70,lastline=81]{src/main/scala/04-WriterMonad.scala}

\end{frame}

\begin{frame}{Changes to output}

\begin{itemize}
  \pause \item When we inject the config into the reader we now get back a writer.
  \pause \item \texttt{writer.run} runs the writer and returns a tuple of logged state and output.
\end{itemize}

\pause

\lstinputlisting[language=Scala,firstline=83,lastline=85]{src/main/scala/04-WriterMonad.scala}

\end{frame}

\begin{frame}{Writer Monad - Results}

\begin{itemize}
\item The appending and passing around of logs is now hidden. Win!
\pause \item The nested monads are very awkward, though. We can still do better.
\end{itemize}

\end{frame}

\section{ReaderWriterT}

\begin{frame}{Introducing Monad Transformers}

\begin{itemize}
\item A way to stack monads on top of each other to feel like one monad.
\pause \item Gets rid of the nesting when mapping / flatmapping
\pause \item No general transformer. Written for each monad.
\pause \item To mimic what we had before, we want these types:
\end{itemize}

\lstinputlisting[language=Scala,firstline=18,lastline=20]{src/main/scala/05-ReaderWriter.scala}

\end{frame}

\begin{frame}{Changes to indented and verifyNewElement}

\begin{itemize}
\item Everything is in PpReaderWriter
\pause \item Note: Kleisli $\equiv$ ReaderT
\end{itemize}

\lstinputlisting[language=Scala,firstline=26,lastline=37]{src/main/scala/05-ReaderWriter.scala}

\end{frame}

\begin{frame}{Changes to stream processor}

\begin{itemize}
\item Interleaving is gone!
\end{itemize}

\lstinputlisting[language=Scala,firstline=52,lastline=66]{src/main/scala/05-ReaderWriter.scala}

\end{frame}

\begin{frame}{ReaderWriter Results}

\begin{itemize}
\item We have our reader and writer enabled program as nice as we can.
\item Lets move on to get rid of the mutable stack!
\end{itemize}

\end{frame}

\section{ReaderWriterState}

\subsection{ReaderWriter + Explicit State passing}

\begin{frame}{Removing the mutable stack}
\begin{itemize}
\item We need to bring the stack into our monad so that we have a completely sealed monad.
\pause \item Lets just try threading the stack through our fold with this structure:
\end{itemize}


\pause

\lstinputlisting[language=Scala,firstline=33,lastline=36]{src/main/scala/06-ReaderWriterCollectingState.scala}

\end{frame}

\begin{frame}{Changes to indentEvent}
\begin{itemize}
\item Each ReaderWriter returns a tuple of the new stack and output
\end{itemize}

\pause

\lstinputlisting[language=Scala,firstline=54,lastline=67]{src/main/scala/06-ReaderWriterCollectingState.scala}

\end{frame}

\begin{frame}{Changes to processor}
\begin{itemize}
\item Fold has to pass the stack through and append output
\end{itemize}

\pause

\lstinputlisting[language=Scala,firstline=70,lastline=82]{src/main/scala/06-ReaderWriterCollectingState.scala}

\end{frame}

\begin{frame}{Explicit State Threading - Results}
\begin{itemize}
\item Passing through the state each step is really annoying
\pause \item Multiple levels of this state would be impractical.
\end{itemize}

\end{frame}

\subsection{ReaderWriterState as a Transformer stack}

\begin{frame}{Introducing the state monad}
\begin{itemize}
\item This monad composes functions, much like reader.
\pause \item State functions are of the form \texttt{State(s => (newState,output)) }
\pause \item We can modify the state for the next thing in the chain.
\pause \item Lets add it to our existing transformer stack in the pretty printer:
\end{itemize}

\pause

\lstinputlisting[language=Scala,firstline=22,lastline=26]{src/main/scala/07-ReaderWriterStateTransformer.scala}

\end{frame}

\begin{frame}{Build our stateful primitives}

Can reuse these anywhere in our program.

\lstinputlisting[language=Scala,firstline=28,lastline=43]{src/main/scala/07-ReaderWriterStateTransformer.scala}
\end{frame}

\begin{frame}{New Indenter}

\lstinputlisting[language=Scala,firstline=45,lastline=51]{src/main/scala/07-ReaderWriterStateTransformer.scala}

\begin{itemize}
\item Our primitives come into play here.
\pause \item Do notation nicely hiding the fact that there are monads at all!
\end{itemize}

\end{frame}

\begin{frame}{New verifyNewElement}

\lstinputlisting[language=Scala,firstline=53,lastline=65]{src/main/scala/07-ReaderWriterStateTransformer.scala}

\begin{itemize}
\item The layers are making this pretty ugly.
\end{itemize}

\end{frame}

\begin{frame}{New indentEvent}

\lstinputlisting[language=Scala,firstline=67,lastline=78]{src/main/scala/07-ReaderWriterStateTransformer.scala}

\begin{itemize}
\item This is very easy to read, and is not too dissimilar from our original.
\end{itemize}

\end{frame}

\begin{frame}{New Fold}

\lstinputlisting[language=Scala,firstline=81,lastline=93]{src/main/scala/07-ReaderWriterStateTransformer.scala}

\begin{itemize}
\item Stack state is gone from our fold. 
\item Stack initialized at end like the config.
\end{itemize}

\end{frame}

\begin{frame}{State Monad - Results}

\begin{itemize}
\item All state is now encapsulated without our monad.
\pause \item Thanks to the primitives, our logic code is clean and 'monad free'
\pause \item The primitives,however, are a bit nasty to create.
\pause \item Because our monads are function based, can't shortcut the layers
very well with point and liftM.
\pause \item Would be nice to flatten this even more.
\end{itemize}

\end{frame}

\subsection{ReaderWriterState from Scalaz}

\begin{frame}{Introducing ReaderWriterState}
\begin{itemize}
\item Instead of being layers of Reader, Writer and State, this is just one layer.
\pause \item Composes functions of the form: \\ \texttt{ReaderWriterState( (r,s) =>
(newLog , output, newState) )}
\pause \item Gives us the same functionality, but easier to construct our
functions.
\pause \item This shortcut exists because RWS stacks are very common and should
be as nice as possible.
\end{itemize}
\end{frame}

\begin{frame}{Primitive Improvements}

\lstinputlisting[language=Scala,firstline=26,lastline=28]{src/main/scala/08-ReaderWriterState.scala}

\lstinputlisting[language=Scala,firstline=47,lastline=58]{src/main/scala/08-ReaderWriterState.scala}

\end{frame}

\section{Conclusions}

\begin{frame}{Conclusions}
\begin{itemize}
\pause \item Our pure code still carries state through the program.
\pause \item All we have done is enforce that side effects must be isolated and
serialised.
\pause \item We're also being explicit about what kind of state a piece of code
depends on.
\pause \item Because the state is restricted and explicit, it is easier to
reason about and is safer.
\pause \item Even though we have these restrictions, our original algorithm
still remains the same.
\pause \item In our imperative code, we were taking risks and burdening
ourselves with flexibility that wasn't even necessary!
\end{itemize}
\end{frame}

\begin{frame}{Three Points I hope I left with you}
\begin{itemize}
\pause \item The motivations for ReaderWriterState and how it is used.
\pause \item Isolation of side effects is a good idea.
\pause \item Types and FP don't get in your way: they keep you honest and help reduce bugs.
\end{itemize}
\end{frame}

\begin{frame}{Further Reading}
\begin{itemize}
\pause \item LYAHFGG
\pause \item FP in Scala
\end{itemize}
\end{frame}

\begin{frame}{The end}
\begin{itemize}
\item Thanks for listening!
\end{itemize}
\end{frame}


\end{document}
